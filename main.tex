
\documentclass[11pt]{article}


\newcommand{\cbsdb}{\textsc{CbsDB}}
\newcommand{\cbs}{\textsc{Cbs}}

%\newcommand{\fullcheck}{\textsc{FullCheck}}
%\newcommand{\autocheck}{\textsc{AutoCheck}}

\newcommand{\fullcheck}{FullSAMC}
\newcommand{\autocheck}{AutoSAMC}
\newcommand{\deepcheck}{DeepSAMC}

\newcommand{\runstat}{\textsc{RunState}}
\newcommand{\runprev}{\textsc{RunPrev}}
\newcommand{\runrec}{\textsc{RunRec}}

\newcommand{\taxdc}{\textsc{TaxDC}}

\newcommand{\ie}{\textit{i.e.}}
\newcommand{\eg}{\textit{e.g.}}
\newcommand{\etc}{\textit{etc.}}

\newcommand{\modist}{\textsc{MoDist}}

\newcommand{\sck}{\textsc{SCk}}


\usepackage{titling}

\usepackage{times}

\usepackage{titlesec}% http://ctan.org/pkg/titlesec
\titleformat{\section}%
  [hang]% <shape>
  {\normalfont\bfseries\Large}% <format>
  {}% <label>
  {0pt}% <sep>
  {}% <before code>
\renewcommand{\thesection}{}% Remove section references...
\renewcommand{\thesubsection}{\arabic{subsection}}%... from subsections

\setlength{\droptitle}{-5.5em} 

\setlength{\topmargin}{-.6in} 
\setlength{\textheight}{9in}
\setlength{\textwidth}{7in} 
\setlength{\headheight}{26pt}
\setlength{\headsep}{9pt} 
\setlength{\oddsidemargin}{-.26in}
\setlength{\evensidemargin}{.25in}

\begin{document}

\title{Research Statement}
\author{Tanakorn Leesatapornwongsa}
\date{\vspace{-1ex} \small{Department of Computer Science, University of
Chicago}}

\maketitle

My research focuses on improving the dependability of distributed and cloud
systems in the aspect of reliability. Large-scale cloud computing systems have
played a major role behind the Information Age, and humanity will soon be highly
relying on cloud computing. Users demand for 24/7 dependability of cloud
computing systems so the systems must be reliable. They must be accessible
anytime and anywhere, and not lose or corrupt users data. Unfulfilled
dependability is costly. Internet service companies collectively lose billions
of dollars in revenue each year from service downtimes. Yet, there are complex
challenges to reach an ideal dependability. Behind cloud computing is a
collection of hundreds of complex systems written in millions of lines of code
that are brittle and prone to failures.

I find that one unsolved reliability problem in cloud systems is
\textit{distributed concurrency bugs (DC bugs)}. DC bugs are caused by
non-deterministic order of distributed events. An example of Distributed events
can be message transmission, which network could non-deterministically re-order
them, or even worse, drop some messages. I also consider machine crashes and
reboots as distributed events, because machine fault is a norm in hardware
infrastructure today. If we consider cloud systems execute multiple complicated
distributed protocols in parallel (\eg, serving users' requests and operating
some background tasks at the same time), possible interleavings of the
distributed events are beyond developers' imagination and there could exist some
interleavings that cannot be handled properly. The buggy distributed
interleavings can cause catastrophic failures such as data loss, data
inconsistencies and downtimes. 

Compared to the countless efforts in combating local concurrency bugs (LC bugs),
bugs happen due to non-determinism of thread scheduling in multi-threaded
software, DC bugs have not received the same amount of attention. I believe it
is time for the dependability community to address this important problem in
systematic and comprehensive manners. And in order to combat DC bugs, I will
pursue three objectives in advancing cloud systems dependability as briefly
shown below.

\if 0
I find that one unsolved reliability problem in cloud systems is distributed
concurrency bugs (DC bugs). DC bugs are caused by non-deterministic order of
distributed events such as message arrivals, machine crashes, and reboots. Buggy
distributed interleavings can cause catastrophic failures such as data loss,
data inconsistencies and downtimes. Compared to the countless efforts in
combating local concurrency bugs (LC bugs), bugs in multi-threaded software, DC
bugs have not received the same amount of attention. I believe it is time for
the dependability community to address this important problem in systematic and
comprehensive manners. And in order to combat DC bugs, I will pursue three
objectives in advancing cloud systems dependability as briefly shown below.
\fi

\begin{enumerate}

\if 0
\item \textbf{Bug study}: Before start a battle, we have to thoroughly know our
enemy, so we can know how to win. Thus in the first step, I start with an
in-depth study of DC bugs and attempt to answer important questions such as
\textit{``What types of DC bugs exist in real world?''} or \textit{``What are
the root causes of DC bugs''}. This study will be an important foundation of
future DC-bug researches in cloud systems community.
\fi

\item \textbf{Bug study}: Before start a battle, we have to thoroughly know our
enemy, so we can plan how to win. Thus in the first step, I started with an
in-depth study of DC bugs to answer important questions that helps guide the
subsequent my research projects and will be an important foundation of future
DC-bug research in cloud systems community.

\item \textbf{Offline testing}: My priority to combat DC bugs is in existing
systems. To focus on these systems, the effective approach is to help
developers unearth DC bugs in these systems. I have built model checking
framework to do offline testing to detect hidden DC bugs. With this framework,
developers will find DC bugs that could lead to severe failures in systems and
fix them before deployment.

\item \textbf{Runtime supports}: Although offline testing can point out most DC
bugs to fix, but it cannot eliminate all, and patches could take weeks to months
to apply into production systems. Thus we still need online/runtime approaches
to complement. I propose three runtime supports that are runtime debugging,
prevention, and recovery. With these supports, we can relieve failures due to DC
bugs in production systems.

\if 0
\item \textbf{DC-bug-free design}: For systems we will develop in the future, I
will advance system designs to allow more concurrency with DC-bug-free
guarantees verified by offline testing and online supports above. And for
hard-to-verify protocols, I will re-design the system with less concurrency.
\fi

\end{enumerate}

The following sections describe in detail my research work grouped by status.

\section*{Current Work}

\subsection{Bug Study}

Many dependability researchers recently employ a \textit{formal bug study}
approach where they analyze hundreds of reported bugs to identify opportunities
for new research, build taxonomies of new problems, test new tools, and for
other purposes. Thus, I start my research here by doing formal buy study.

\subsubsection*{Initial work: Cloud Bug Study}

Our research group and I have performed the largest bugs study in six important
Apache cloud infrastructures including Cassandra, Flume, Hadoop MapReduce,
HBase, HDFS, and ZooKeeper (\textit{SoCC '14}). We reviewed in total 21,399
submitted issues within a three-year period (2011-2014) in Apache bug
reposistories. Among these issues, we perform a deep analysis of 3655 ``vital''
issues (\ie, real issues affecting deployments) with a set of detailed
classifications. This work led our group to several interesting dependability
research questions, however, it did not dissect DB bugs. 

\subsubsection*{Ongoing work: DC Bug Taxonoy} 

While there were many LC-bug study, I am not aware of any large-scale study of
DC bugs. A recent paper from Tsinghua University and Microsoft Research only
studied five DC bugs in MapReduce systems. Thus to fill the void, our group and
I have created the largest and most comprehensive taxonomy of 104 real-world DC
bugs (named TaxDC) from Cassandra, HBase, Hadoop MapReduce/Yarn, and ZooKeeper.
TaxDC contains in-depth characteristics of DC bugs, stored in the form of 2083
classification labels and 4528 lines of re-enumerated steps to the bugs that we
manually added. Motivated by the availability of bug benchmarks for LC bugs, we
will release TaxDC as a large-scale DC bugs benchmark.

With TaxDC, I can answer important questions such as: How often DC bugs are
reported from real deployments? What types of DC bugs exist in real world?  What
are the root causes of DC bugs (out-of-order messages, failures, \etc)? Can
existing LC-bug-detection tools applicable for DC bugs? How do developers fix DC
bugs (by adding locks, states, \etc)? What are the inputs/triggering conditions?
What are the minimum number of distributed events needed to trigger the bugs
(how many messages to re-order, failures to inject, \etc)? What errors/effects
(specification violations) are caused by DC bugs? (NPE, deadlock, data loss,
state inconsistency, performance problems, \etc)? How do propagation chains form
from the root causes to errors? The answers to these questions will guide the
subsequent my research projects.

\subsection{Offline Testing}

After I have understood the characteristics of DC bugs, I started unearth hidden
bugs in current cloud infrastructures by using offline testing. One powerful
method for discovering hidden DC bugs is the use of an implementation-level
distributed system model checker (dmck). By re-ordering non-deterministic
distributed events, a dmck can discover buggy interleavings that lead to DC
bugs.

The last eight years have seen a rise of dmcks. One big challenge faced by a
dmck is the state-space explosion problem (\ie, there are too many distributed
events to re-order, and re-ordering every permutation cannot be done in
reasonable time). To address this, existing dmcks adopt a random walk or basic
reduction techniques such as dynamic partial order reduction (DPOR). Despite
these early successes, existing approaches are still impractical to unearth many
real-world DC bugs due to complex non-determinism that could happen (\eg, an
additional crash during current crash recovery), so we advanced the state of the
art of dmck to make it possible to test real-world cloud systems.

\subsubsection*{Initial work: Semantic-Aware Model Checking} 

To start with, I specifically addressed two limitations of existing dmcks.
First, existing dmcks treat every target system as a complete black box, and
therefore perform unnecessary reorderings of distributed events that would lead
to the same explored states (\ie, redundant executions). Second, they do not
incorporate complex multiple fault events (\eg, crashes, reboots) into their
exploration strategies, as such inclusion would exacerbate the state-space
explosion problem.

To address these limitations, I built Semantic-Aware Model Checker (SAMC)
(\textit{OSDI '14, ISSTA '15}), a white-box model checker that takes semantic
knowledge of how distributed events (specifically, messages, crashes, and
reboots) are processed by the target system and incorporates that information in
reduction policies. The policies are based on sound reduction techniques such as
DPOR and symmetry. The policies tell SAMC not to re-order some pairs of events
such as message-message pairs or crash-message pairs, yet preserves soundness,
because those cut out re-orderings are redundant.

With SAMC, I was able to unearth 12 old bugs in 3 cloud systems involving 30-120
distributed events and multiple crashes and reboots. Some of these bugs cannot
be unearthed by non-SAMC approaches, even after two days. SAMC can find the bugs
271x (33x on average) compared to state-of-the-art techniques. And I was able to
find a few new bugs in Hadoop MapReduce and ZooKeeper.

\subsubsection*{Ongoing work: Full Check Model Checking} 

Although SAMC can significantly mitigate state-space explosion, I still find two
major gaps between SAMC and real-world DC bugs. First, SAMC reorders messages by
default and injects crashes and reboots, but it does not control the timings of
all types of events necessary to unearth DC bugs. For example, SAMC does not
intercept local computation, does not exercise timeouts, and does not include
other faults rather than crashes and reboots such as untimely disk faults.

Second, controlling all necessary events is technically doable, but it will
``blow up'' the exploration space. The use of semantic relationships between
multiple events such as message-message and crash-message semantics in SAMC can
remove redundant re-orderings. However, more innovations are needed to devise
fast exploration strategies that leverage semantic relationships among all
necessary events.

To address the incompleteness of SAMC, my colleagues and I are building
\fullcheck, a dmck that intercepts all types of necessary events to unearth
real-world DC bugs, but will do so in a fast and scalable manner with the
incorporation of semantic relationships between the events.

\subsubsection*{Ongoing work: Automatically Generated Reduction Strategies} 

As we leverage domain-specific semantic information into reduction strategies,
this process is done by hand that means it requires developers manually extract
and incorporate the semantic knowledge and write the corresponding reduction
policies. This manual process is based on high-level human understanding of the
codebase, which can potentially miss important patterns due to human errors, and
breaks soundness, which then makes us skip buggy orderings.

To address the unsoundness of SAMC and \fullcheck, and the developers' burden in
manually writing semantic-based reduction strategies, my colleagues and I are
evolving it into \autocheck, a dmck that automatically and soundly extracts
complete semantic knowledge into reduction strategies with the help of program
analysis.

To do so, we combine symbolic execution and dmck. While others have used
symbolic execution with model checking for LC bugs, \autocheck will be the first
case for implementation-level dmck. 

\section*{Future Work}

\subsection{Runtime Support}

While my current work is on an offline approach, my future plan includes runtime
supports to combat DC bugs. I plan to build three runtime supports: (1) a
runtime monitoring and statistical debugging support; (2) a runtime DC-bug
prevention layer; and (3) a runtime checkpoint-rollback recovery mechanism.

\subsubsection*{Runtime DC-Bug Statistical Debugging}

Statistical debugging has been used effectively for detecting performance
anomalies in distributed systems and for comparing success- and failure-run
traces to help identify LC-bug triggering conditions Such an approach can also
be powerful to unearth DC bugs uncaught in offline approaches and useful to
developers who want to debug customer deployments. 

Thus the first step of runtime supports, I will build a runtime monitoring and
statistical debugging support to catch DC bugs in live deployments, specifically
by pinpointing the triggering conditions.

\subsubsection*{Runtime DC-Bug Prevention}

Applying patches to deployments can take weeks to months. Ideally, we want to
alter the timings of triggering conditions to prevent DC-bug-induced failures
before the patches are applied. Altering runtime behaviors has proven to be
effective for LC bugs, unfortunately their detailed approaches are not directly
applicable for DC bugs; LC-bug-prevention approaches target atomicity violations
that are rare in distributed systems. Fortunately, our study shows that many DC
bugs can be prevented by altering the timings of a few set of events or by
retrying or discarding some messages, creating significant opportunities for
automated DC bug prevention.

I will develop a runtime prevention layer built around each node of the target
system that manipulates event timing and processing to prevent DC-bug-induced
failures to manifest. Developers can program the runtime prevention layer on the
fly to apply simple fixes.

\subsubsection*{Runtime DC-Bug Recovery}

I envision that runtime prevention can prevent most DC bugs, but it has two
limitations. First, as it depends on input specification, it cannot guarantee
100\% bug coverage. Second, if DC-bug-induced failures happen at critical nodes
(\eg, master node), fail-stopping the node is not an ideal solution. Thus, we
want to transparently recover from DC bugs in production runs, minimizing
user-visible failures. Similar approaches have been successfully used for LC
bugs; when an LC-bug-induced failure is detected, the system can rollback to
the last good checkpoint and re-run the computation hoping the
non-deterministic buggy interleaving will not repeat.

Thus I will apply the same idea of LC-bug recovery to build a runtime recovery
mechanism that performs checkpoint and rollback to mask DC-bug-induced failures.

%\subsection{DC-Bug-Free Design}

\end{document}
