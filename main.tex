
\documentclass[11pt]{article}


\newcommand{\cbsdb}{\textsc{CbsDB}}
\newcommand{\cbs}{\textsc{Cbs}}

%\newcommand{\fullcheck}{\textsc{FullCheck}}
%\newcommand{\autocheck}{\textsc{AutoCheck}}

\newcommand{\fullcheck}{FullSAMC}
\newcommand{\autocheck}{AutoSAMC}
\newcommand{\deepcheck}{DeepSAMC}

\newcommand{\runstat}{\textsc{RunState}}
\newcommand{\runprev}{\textsc{RunPrev}}
\newcommand{\runrec}{\textsc{RunRec}}

\newcommand{\taxdc}{\textsc{TaxDC}}

\newcommand{\ie}{\textit{i.e.}}
\newcommand{\eg}{\textit{e.g.}}
\newcommand{\etc}{\textit{etc.}}

\newcommand{\modist}{\textsc{MoDist}}

\newcommand{\sck}{\textsc{SCk}}


\usepackage{titling}

\usepackage{times}

\if 0
\usepackage{titlesec}% http://ctan.org/pkg/titlesec
\titleformat{\section}%
  [hang]% <shape>
  {\normalfont\bfseries\Large}% <format>
  {}% <label>
  {0pt}% <sep>
  {}% <before code>
\renewcommand{\thesection}{}% Remove section references...
\renewcommand{\thesubsection}{\arabic{subsection}}%... from subsections
\fi

\setlength{\droptitle}{-5.5em} 

\setlength{\topmargin}{-.6in} 
\setlength{\textheight}{9in}
\setlength{\textwidth}{7in} 
\setlength{\headheight}{26pt}
\setlength{\headsep}{9pt} 
\setlength{\oddsidemargin}{-.26in}
\setlength{\evensidemargin}{.25in}

\begin{document}

\title{Research Summary}
\author{Tanakorn Leesatapornwongsa}
\date{\vspace{-1ex} \small{Department of Computer Science, University of
Chicago}}

\maketitle

\vspace{-1ex} 
My research focuses on improving the dependability of cloud-scale distributed
systems such as scale-out storage systems, distributed computing frameworks,
and cluster management services. Unfulfilled dependability is costly. Internet
service companies collectively lose billions of dollars in revenue each year
from service downtimes. Yet, there are complex challenges to reach an ideal
dependability. 

%Behind cloud computing is a collection of hundreds of complex
%systems written in millions of lines of code that are brittle and prone to
%failures.

I find that one unsolved dependability problem in cloud systems is
\textbf{\textit{distributed concurrency bugs (DC bugs)}}. DC bugs are caused by
non-deterministic order of distributed events such as message arrivals, faults,
and reboots. Cloud systems execute multiple complicated distributed protocols
concurrently (\eg, serving users' requests, operating some background tasks,
and combined with untimely hardware failues). The possible interleavings of the
distributed events are beyond developers' imagination and some interleavings
might not be handled properly. The buggy interleavings can cause catastrophic
failures such as data loss, data inconsistencies and downtimes. 

%Compared to the countless efforts in combating local concurrency bugs (LC bugs),
%which happen due to non-determinism of thread scheduling in multi-threaded
%software, DC bugs have not received the same amount of attention. I believe it
%is time for the dependability community to address this important problem in
%systematic and comprehensive manners. 

To combat DC bugs, I establish my research to (1) do \textbf{\textit{formal bug
study}} and (2) unearth DC bugs by \textbf{\textit{model checking}}. The
following sections explain my research in detail and how it is applicable to
Facebook.

\section{Bug Study}

Bug or failure studies can significantly guide many aspects of dependability
research. Many dependability researchers recently employ formal studies on
bugs/failures such as a large-scale study of flash memory failures from
Facebook that can identify opportunities for new research. For distributed
system bug, there were a few studies on them but the works did not dissect DC
bugs, and I am not aware of any large-scale study of DC bugs. 

To fill the void, our group and I have created the largest and most
comprehensive taxonomy of 104 real-world DC bugs (named \taxdc) from Cassandra,
HBase, Hadoop MapReduce/Yarn, and ZooKeeper. \taxdc\ contains in-depth
characteristics of DC bugs and re-enumerated steps to the bugs that we manually
added. And it helps me answer important questions such as: What types of DC
bugs exist in real world? What are the root causes of DC bugs?  How do
developers fix DC bugs? How do propagation chains form from the root causes to
errors? The answers for these questions help guide the subsequent of my
research.

%, stored in the form of 2083 classification labels
%and 4528 lines of re-enumerated steps to the bugs that we manually added. 

%With \taxdc, I can answer important questions such as: What types of DC bugs
%exist in real world? What are the root causes of DC bugs?  How do developers
%fix DC bugs? How do propagation chains form from the root causes to errors? The
%answers for these questions help guide the subsequent of my research.

\section{Model Checking}

One powerful method for discovering hidden DC bugs is the use of an
implementation-level distributed system model checker (dmck). Dmcks discover
hidden DC bugs by re-ordering all non-deterministic distributed events.
However, one big challenge faced by dmcks is the state-space explosion problem
(\ie, there are too many distributed events to re-order). In order to address
this, existing dmcks adopt basic reduction techniques such as dynamic partial
order reduction (DPOR). Despite these early successes, existing approaches are
still impractical to unearth real-world bugs, so I am advancing the state of
the art of dmck to combat DC bugs, which I describe below.

\subsection{Semantic-Aware Model Checking (Initial Work)} 

I started my work by specifically addressing two limitations of existing dmcks.
First, existing dmcks treat target system as a complete black box, and perform
unnecessary reorderings of distributed events that would lead to the same
explored states (\ie, redundant executions). Second, they do not incorporate
complex multiple fault events (\eg, crashes, reboots) into their checks, as
such inclusion would exacerbate the state-space explosion problem.

To address these limitations, I introduce Semantic-Aware Model Checking (SAMC)
(\textit{OSDI '14, ISSTA '15}), a white-box model checking approach that takes
\textit{semantic knowledge} of how distributed events (specifically, messages,
crashes, and reboots) are processed and incorporates that information to create
reduction policies. The policies are based on sound reduction techniques such
as DPOR and symmetry. The policies tell model checker not to re-order some
pairs of events such as message-message pairs or crash-message pairs, yet
preserves soundness, because those cut out re-orderings are redundant.

With SAMC, I was able to unearth 12 old bugs in Cassandra, Hadoop MapReduce,
and ZooKeeper. Some bugs cannot be unearthed by non-SAMC approaches, even after
two days. SAMC can find the bugs up to 271x (33x on average) faster compared to
state-of-the-art techniques. And I was able to find a few new bugs in these
systems.

\if 0
\subsection{Full Semantic-Aware Model Checking} 

Although SAMC can significantly mitigate state-space explosion, I still find two
major gaps between SAMC and real-world DC bugs. First, SAMC reorders messages by
default and injects crashes and reboots, but it does not control the timings of
all types of events necessary to unearth DC bugs. For example, SAMC does not
intercept local computation, does not exercise timeouts, and does not include
other faults rather than crashes and reboots such as untimely disk faults.

Second, controlling all necessary events is technically doable, but it will
``blow up'' the exploration space. The use of semantic relationships between
multiple events such as message-message and crash-message semantics in SAMC can
remove redundant re-orderings. However, more innovations are needed to devise
fast exploration strategies that leverage semantic relationships among all
necessary events.

To address the incompleteness of SAMC, my colleagues and I are building
\fullcheck, a dmck that intercepts all types of necessary events to unearth
real-world DC bugs, but will do so in a fast and scalable manner with the
incorporation of semantic relationships between the events.
\fi

\subsection{Automated Semantic-Aware Model Checking (Ongoing Work)} 

So far, as we leverage domain-specific semantic information into reduction
strategies, we (or the developers) must manually extract and incorporate the
semantic knowledge and write the corresponding reduction policies. This manual
process is based on high-level human understanding of the codebase, which can
potentially miss important re-orderings due to human errors, and breaks
soundness, which could leave DC bugs unearthing.

To address the unsoundness of SAMC, and the developers' burden in manually
writing semantic-based reduction strategies, I am developing \autocheck, a dmck
that automatically and soundly extracts semantic knowledge into reduction
strategies with the help of static analysis. I am combining symbolic execution
and dmck. \autocheck\ will be the first case for implementation-level dmck that
adopts symbolic execution. 

\subsection{Deep Distributed System Model Checking (Future Work)}

Execution paths to DC bugs often requires complex input preconditions such as
multiple faults, reboots, and protocol initiations. From my study, more than
60\% of DC bugs require more than one protocol initiations, and 29\% of DC bugs
arise within buggy interactions between foreground and background protocols.
This again highlights the complexity of fully complete systems. If we do not
include the complex preconditions, the bugs will not surface.

To address this complexity, I will construct \deepcheck, a dmck with a backward
static analysis tool that is capable of searching the necessary input
preconditions to cover unreachable paths. The concept of \deepcheck\ is the dmck
will run with a limited input precondition. Then, \deepcheck\ will analyze which
code path is not reachable given the limited input. It will perform a backward
analysis to search for input preconditions to the path. As a result, this
backward analysis will provide the sequence of input preconditions that cover
more complex scenarios.

\section{Research Applicability}

One factor behind the success of Facebook is its robust large-scale data
centers. Facebook has developed a number of cloud-scale systems such as
Cassandra, the popular open-source scalable database, Haystack and f4, optimized
object storages for photo sharing, and Unicorn, social-graph indexing system.
Facebook's systems were written in gigantic size of codebase and if not
thoroughly tested, there could be corner-case bugs hiding there.

Facebook has developed tools to detect bugs such as Facebook Infer, static
analyzer for mobile apps. For cloud systems, a Facebook engineer collaborated
with academic institutes to build \fad, fault injection framework for
distributed systems which inject fault to systems in order to catch DC bugs due
to the fault timing. However, \fad\ do not address untimely message arrivals,
they can discover just a subset of DC bugs.

As shown in SAMC, I successfully model checked Cassandra and the others. My
model checking framework is also applicable to other systems as well. To
integrate SAMC with existing systems, it does not need to modify the current
codebase, so it does not interrupt ongoing development processes. As I envision
my future works to make system verfication to become automated (\ie, \autocheck\
and \deepcheck), if Facebook adopts them, they will help improve dependability
of Facebook's systems and reduce the cost of system verification process at
Facebook.

\end{document}

