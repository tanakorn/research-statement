
\documentclass[11pt]{article}


\newcommand{\cbsdb}{\textsc{CbsDB}}
\newcommand{\cbs}{\textsc{Cbs}}

%\newcommand{\fullcheck}{\textsc{FullCheck}}
%\newcommand{\autocheck}{\textsc{AutoCheck}}

\newcommand{\fullcheck}{FullSAMC}
\newcommand{\autocheck}{AutoSAMC}
\newcommand{\deepcheck}{DeepSAMC}

\newcommand{\runstat}{\textsc{RunState}}
\newcommand{\runprev}{\textsc{RunPrev}}
\newcommand{\runrec}{\textsc{RunRec}}

\newcommand{\taxdc}{\textsc{TaxDC}}

\newcommand{\ie}{\textit{i.e.}}
\newcommand{\eg}{\textit{e.g.}}
\newcommand{\etc}{\textit{etc.}}

\newcommand{\modist}{\textsc{MoDist}}

\newcommand{\sck}{\textsc{SCk}}


\usepackage{titling}

\usepackage{times}

\if 0
\usepackage{titlesec}% http://ctan.org/pkg/titlesec
\titleformat{\section}%
  [hang]% <shape>
  {\normalfont\bfseries\Large}% <format>
  {}% <label>
  {0pt}% <sep>
  {}% <before code>
\renewcommand{\thesection}{}% Remove section references...
\renewcommand{\thesubsection}{\arabic{subsection}}%... from subsections
\fi

\setlength{\droptitle}{-5.5em} 

\setlength{\topmargin}{-.6in} 
\setlength{\textheight}{9in}
\setlength{\textwidth}{7in} 
\setlength{\headheight}{26pt}
\setlength{\headsep}{9pt} 
\setlength{\oddsidemargin}{-.26in}
\setlength{\evensidemargin}{.25in}

\begin{document}

\title{Research Summary}
\author{Tanakorn Leesatapornwongsa}
\date{\vspace{-1ex} \small{Department of Computer Science, University of
Chicago}}

\maketitle

\vspace{-1ex} 
My research focuses on improving the dependability of cloud-scale distributed
systems such as scale-out storage systems, distributed computing frameworks, and
cluster management services. Unfulfilled dependability is costly. Internet
service companies collectively lose billions of dollars in revenue each year
from service downtimes. Yet, there are complex challenges to reach an ideal
dependability. Behind cloud computing is a collection of hundreds of complex
systems written in millions of lines of code that are brittle and prone to
failures.

I find that one unsolved dependability problem in cloud systems is
\textbf{\textit{distributed concurrency bugs (DC bugs)}}. DC bugs are caused by
non-deterministic order of distributed events such as message arrivals, faults,
and reboots. Cloud systems execute multiple complicated distributed protocols
concurrently (\eg, serving users' requests, operating some background tasks at
the same time, and combined with untimely hardware failues). The possible
interleavings of the distributed events are beyond developers' imagination and
some interleavings might not be handled properly. The buggy interleavings can
cause catastrophic failures such as data loss, data inconsistencies and
downtimes. 

Compared to the countless efforts in combating local concurrency bugs (LC bugs),
which happen due to non-determinism of thread scheduling in multi-threaded
software, DC bugs have not received the same amount of attention. I believe it
is time for the dependability community to address this important problem in
systematic and comprehensive manners. To combat DC bugs, I establish my research
to (1) do \textit{formal bug study} and (2) build \textit{model checking
framework} to unearth DC bugs in systems. The following sections explain my
research in detail and how it is applicable to Facebook.

\section{Bug Study}

Bug or failure studies can significantly guide many aspects of dependability
research. Many dependability researchers recently employ formal studies on
bugs/failures such as Facebook have published a large-scale study of flash
memory failures which they can identify opportunities for new research. For
distributed system bug, there are a few studies on them but the works did not
dissect DC bugs, and I am not aware of any large-scale study of DC bugs. 

To fill the void, our group and I have created the largest and most
comprehensive taxonomy of 104 real-world DC bugs (named \taxdc) from Cassandra,
HBase, Hadoop MapReduce/Yarn, and ZooKeeper. \taxdc\ contains in-depth
characteristics of DC bugs, stored in the form of 2083 classification labels
and 4528 lines of re-enumerated steps to the bugs that we manually added. 

With \taxdc, I can answer important questions such as: What types of DC bugs
exist in real world? What are the root causes of DC bugs (out-of-order
messages, hardware failures, \etc)? Can existing LC-bug-detection tools
applicable for DC bugs? How do developers fix DC bugs (by adding locks, states,
\etc)? How do propagation chains form from the root causes to errors? The
answers for these questions guide the subsequent of my research.

\section{Model Checking}

One powerful method for discovering hidden DC bugs is the use of an
implementation-level distributed system model checker (dmck). Dmcks discover
hidden DC bugs by re-ordering all non-deterministic distributed events.
However, one big challenge faced by a dmck is the state-space explosion problem
(\ie, there are too many distributed events to re-order). In order to address
this, existing dmcks adopt basic reduction techniques such as dynamic partial
order reduction (DPOR). Despite these early successes, existing approaches are
still impractical to unearth real-world bugs, so I am advancing the state of the
art of dmck, which I describe below.

\subsection{Semantic-Aware Model Checking} 

I started my work by specifically addressing two limitations of existing dmcks.
First, existing dmcks treat every target system as a complete black box, and
therefore perform unnecessary reorderings of distributed events that would lead
to the same explored states (\ie, redundant executions). Second, they do not
incorporate complex multiple fault events (\eg, crashes, reboots) into their
exploration strategies, as such inclusion would exacerbate the state-space
explosion problem.

To address these limitations, I built Semantic-Aware Model Checker (SAMC)
(\textit{OSDI '14, ISSTA '15}), a white-box model checker that takes semantic
knowledge of how distributed events (specifically, messages, crashes, and
reboots) are processed by the target system and incorporates that information in
reduction policies. The policies are based on sound reduction techniques such as
DPOR and symmetry. The policies tell SAMC not to re-order some pairs of events
such as message-message pairs or crash-message pairs, yet preserves soundness,
because those cut out re-orderings are redundant.

With SAMC, I was able to unearth 12 old bugs in 3 cloud systems including
Cassandra, Hadoop MapReduce, and ZooKeeper, involving 30-120 distributed events
and multiple crashes and reboots. Some of these bugs cannot be unearthed by
non-SAMC approaches, even after two days. SAMC can find the bugs up to 271x (33x
on average) faster compared to state-of-the-art techniques. And I was able to
find a few new bugs in Hadoop MapReduce and ZooKeeper.

\subsection{Full Semantic-Aware Model Checking} 

Although SAMC can significantly mitigate state-space explosion, I still find two
major gaps between SAMC and real-world DC bugs. First, SAMC reorders messages by
default and injects crashes and reboots, but it does not control the timings of
all types of events necessary to unearth DC bugs. For example, SAMC does not
intercept local computation, does not exercise timeouts, and does not include
other faults rather than crashes and reboots such as untimely disk faults.

Second, controlling all necessary events is technically doable, but it will
``blow up'' the exploration space. The use of semantic relationships between
multiple events such as message-message and crash-message semantics in SAMC can
remove redundant re-orderings. However, more innovations are needed to devise
fast exploration strategies that leverage semantic relationships among all
necessary events.

To address the incompleteness of SAMC, my colleagues and I are building
\fullcheck, a dmck that intercepts all types of necessary events to unearth
real-world DC bugs, but will do so in a fast and scalable manner with the
incorporation of semantic relationships between the events.

\subsection{Automated Semantic-Aware Model Checking} 

So far, as we leverage domain-specific semantic information into reduction
strategies, we (or the developers) must manually extract and incorporate the
semantic knowledge and write the corresponding reduction policies. This manual
process is based on high-level human understanding of the codebase, which can
potentially miss important re-orderings due to human errors, and breaks
soundness, which could leave DC bugs unearthing.

To address the unsoundness of SAMC, and the developers' burden in manually
writing semantic-based reduction strategies, I am evolving it into \autocheck, a
dmck that automatically and soundly extracts complete semantic knowledge into
reduction strategies with the help of program analysis. To do so, we combine
symbolic execution and dmck. While others have used symbolic execution with
model checking for LC bugs, \autocheck\ will be the first case for
implementation-level dmck. 

\subsection{Deep Distributed System Model Checking}

The path to the triggering conditions often requires complex input preconditions
such as multiple faults, reboots, and protocol initiations. When I measure the
number of protocol initiations to unearth each DC bug, I find more than 60\% of
DC bugs require more than one protocol initiations. And I find 29\% of DC bugs
arise within buggy interactions between foreground and background protocols.
This again highlights the complexity of fully complete systems. If we do not
include crashes, reboots and a client write, the bugs will not surface.

To address this complexity, I will construct \deepcheck, a dmck with a backward
static analysis tool that is capable of searching the necessary input
preconditions to cover unreachable paths. The concept of \deepcheck\ is the dmck
will run with a limited input precondition. Then, \deepcheck\ will analyze which
code path is not reachable given the limited input. It will perform a backward
analysis to search for input preconditions to the path. As a result, this
backward analysis will provide the sequence of input preconditions that cover
more complex scenarios.

\section{Applicability}

One factor behind the success of Facebook, the largest online social network in
the world, is its robust large-scale data centers. Facebook has developed a
number of cloud-scale systems to deal with big data from its one billion users
such as Cassandra, one of the most popular open-source scalable database,
Haystack and f4, optimized object storages for photo sharing, and Unicorn
social-graph indexing system. These systems were written in millions lines of
code and if not thoroughly tested, there could be corner-case bugs hiding in
there.

Facebook has been developing tools to detect bugs such as \textit{Facebook
Infer}, static analyzer for Android and iOS apps. For cloud systems, a Facebook
engineer collaborated with academic institutes to build \fad, fault injection
framework for distributed systems which inject fault to systems in order to
catch DC bugs due to the fault timing. However, \fad\ do not address untimely
message arrival, they can unearth just a subset of DC bugs.

%As the largest online social network in the world, Facebook has over one billion
%monthly active users, and needs to deal with \textit{big data}. To address the 

As shown in my initial SAMC work, I successfully model checked Cassandra, Hadoop
MapReduce, and ZooKeeper. The model checking framework I built is applicable to
various types of systems, and it is also applicable to other systems as well,
including Facebook's. The technologies I used also allow me to integrate SAMC to
existing systems without modifying current codebase, so it helps improve
dependability of currents systems without interrupting any development
processes. And as I envision my future works to make software verfication to
become more and more automated (\ie, \autocheck\ and \deepcheck). If Facebook
adopts them, this will reduce the cost of verification process at Facebook.

%so if Facebook adopts it, integrating it to
%current Facebook's systems will improve dependability without interrupting any
%development processes.

%Through my model checking research, the works are applicable to existing
%cloud-scale distributed systems. As shown in my initial SAMC work, I
%successfully model checked Cassandra, Hadoop MapReduce, and ZooKeeper. The
%framework can be 

%The framework I have built can be integrated to
%existing systems without modifying current codebase. If we integrate it to
%Facebook's systems, we can improve their dependability.

\end{document}

