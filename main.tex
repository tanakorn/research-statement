
\documentclass[11pt]{article}


\newcommand{\cbsdb}{\textsc{CbsDB}}
\newcommand{\cbs}{\textsc{Cbs}}

%\newcommand{\fullcheck}{\textsc{FullCheck}}
%\newcommand{\autocheck}{\textsc{AutoCheck}}

\newcommand{\fullcheck}{FullSAMC}
\newcommand{\autocheck}{AutoSAMC}
\newcommand{\deepcheck}{DeepSAMC}

\newcommand{\runstat}{\textsc{RunState}}
\newcommand{\runprev}{\textsc{RunPrev}}
\newcommand{\runrec}{\textsc{RunRec}}

\newcommand{\taxdc}{\textsc{TaxDC}}

\newcommand{\ie}{\textit{i.e.}}
\newcommand{\eg}{\textit{e.g.}}
\newcommand{\etc}{\textit{etc.}}

\newcommand{\modist}{\textsc{MoDist}}

\newcommand{\sck}{\textsc{SCk}}


\usepackage{titling}

\usepackage{times}

\usepackage{setspace}
%\doublespacing

\setlength{\droptitle}{-5.5em} 

\setlength{\topmargin}{-.6in} 
\setlength{\textheight}{9in}
\setlength{\textwidth}{7in} 
\setlength{\headheight}{26pt}
\setlength{\headsep}{9pt} 
\setlength{\oddsidemargin}{-.26in}
\setlength{\evensidemargin}{.25in}

\begin{document}

\title{Research Statement}
\author{Tanakorn Leesatapornwongsa}
\date{\vspace{-1ex} \small{Department of Computer Science, University of
Chicago}}

\maketitle

My research focuses on improving the dependability of cloud-scale distributed
systems such as scale-out storage systems, distributed computing frameworks,
synchronization services, and cluster management services. Users demand for
24/7 dependability of cloud computing systems so the systems must be reliable.
They must be accessible anytime and anywhere, and not lose or corrupt users
data. Unfulfilled dependability is costly. Internet service companies
collectively lose billions of dollars in revenue each year from service
downtimes. Yet, there are complex challenges to reach an ideal dependability.
Behind cloud computing is a collection of hundreds of complex systems written in
millions of lines of code that are brittle and prone to failures.

I find that one unsolved reliability problem in cloud systems is
\textbf{\textit{distributed concurrency bugs (DC bugs)}}. DC bugs are caused by
non-deterministic order of distributed events such as message arrivals, faults,
and reboots. Cloud systems execute multiple complicated distributed protocols
concurrently (\eg, serving users' requests, operating some background tasks at
the same time, and combined with untimely hardware failures). The possible
interleavings of the distributed events are beyond developers' imagination and
some interleavings might not be handled properly. The buggy
distributed interleavings can cause catastrophic failures such as data loss,
data inconsistencies and downtimes. 

Compared to the countless efforts in combating \textbf{\textit{local concurrency
bugs (LC bugs)}}, which happen due to non-determinism of thread scheduling in
multi-threaded software, DC bugs have not received the same amount of attention.
I believe it is time for the dependability community to address this important
problem in systematic and comprehensive manners.  To combat DC bugs, I establish
my research in (1) \textbf{\textit{formal bug studies}} and (2)
\textbf{\textit{distributed system model checking}}. The following sections
explain my research in detail.

\if 0
\begin{enumerate}

\item \textbf{Formal bug study}: For the first step, I started with an in-depth
study of DC bugs to answer important questions that helps guide the subsequent
my research projects and will be an important foundation of future DC-bug
research in cloud systems community.

\item \textbf{Distributed system model checking}: To combat DC bugs, my focus is
on existing systems. The effective approach for these systems is to help
developers unearth DC bugs. I have built model checking framework to detect
hidden DC bugs. 

\end{enumerate}

The following sections describe in detail my research work grouped by status.
\fi

\section{Bug Study}

Bug or failure studies can significantly guide many aspects of dependability
research. Many dependability researchers recently employ formal studies on bugs
and failures such as the studies on large-scale system bugs/failures from
Microsoft \cite{Guo+13-CureIsWorse, Li+13-ScopeBugStudy}. These studies can
identify opportunities for new research, build taxonomies of new problems, test
new tools, and for other purposes. I started my research here by doing formal
bug study to gain foundations of combating DC bugs.

\subsection{Cloud Bug Study}

As an initiative, our group and I have performed the largest bugs study in six
important Apache cloud infrastructures including Cassandra, Flume, Hadoop
MapReduce, HBase, HDFS, and ZooKeeper \cite{Gunawi+14-Cbs}. We
reviewed in total 21,399 submitted issues within a three-year period (2011-2014)
in Apache bug reposistories. Among these issues, we perform a deep analysis of
3655 ``vital'' issues (\ie, real issues affecting deployments) with a set of
detailed classifications. This work led our group to several interesting
dependability research questions, and was the main source of my DC-bug taxonomy
work.

\subsection{DC Bug Taxonoy} 

While there were many LC-bug study, I am not aware of any large-scale study of
DC bugs. A recent study from Microsoft analyzed the effect of distributed
concurrency on workload and only studied five DC bugs in MapReduce systems
\cite{Xiao+14-NonDetMR}. To fill the void, I as one of the project leaders,
have created the largest and most comprehensive taxonomy of 104 real-world DC
bugs (named \taxdc) from Cassandra, HBase, Hadoop MapReduce/Yarn, and ZooKeeper
\cite{Gunawi+16-TaxDc-Appear}. \taxdc\ contains in-depth characteristics of DC
bugs, stored in the form of 2083 classification labels and 4528 lines of
re-enumerated steps to the bugs that we manually added. Motivated by the
availability of bug benchmarks for LC bugs, we will release \taxdc\ as a
large-scale DC bugs benchmark.

With \taxdc\, I can answer important questions such as: How often DC bugs are
reported from real deployments? What types of DC bugs exist in real world?
What are the root causes of DC bugs (out-of-order messages, failures, \etc)?
Can existing LC-bug-detection tools applicable for DC bugs? How do developers
fix DC bugs (by adding locks, states, \etc)? What are the inputs/triggering
conditions?  What are the minimum number of distributed events needed to
trigger the bugs (how many messages to re-order, failures to inject, \etc)?
What errors/effects (specification violations) are caused by DC bugs (deadlock,
data loss, state inconsistency, performance problems, \etc)? How do propagation
chains form from the root causes to errors? The answers to these questions will
guide the subsequent my research projects.

\section{Distributed System Model Checking}

\if 0
After gaining the understanding of the characteristics of DC bugs, I started
unearth hidden bugs in current cloud systems by \textit{implementation-level
distributed system model checker} (\textbf{dmck}). Dmck is a powerful method for
discovering hidden DC bugs. By re-ordering non-deterministic distributed events,
a dmck can discover buggy interleavings that lead to DC bugs.
\fi

One powerful method for discovering hidden DC bugs is the use of an
\textit{implementation-level distributed system model checker} (\textbf{dmck}).
By re-ordering non-deterministic distributed events, a dmck can discover buggy
interleavings that lead to DC bugs. The last eight years have seen a rise of
dmcks such as MaceMC \cite{Killian+07-LifeDeathMaceMC}, \modist\
\cite{Yang+09-Modist}, or Demeter \cite{Guo+11-Demeter}. One big challenge faced
by a dmck is the state-space explosion problem (\ie, there are too many
distributed events to re-order). To address this, existing dmcks adopt a random
walk or basic reduction techniques such as dynamic partial order reduction
(DPOR). Despite these early successes, existing approaches cannot unearth many
real-world DC bugs, so I am advancing the state of the art of dmck to combat DC
bugs, which I describe below.

\subsection{Semantic-Aware Model Checking (Initial Work)} 

I started my work by specifically addressing two limitations of existing dmcks.
First, existing dmcks treat every target system as a complete \textit{black
box}, and therefore perform unnecessary reorderings of distributed events that
would lead to the same explored states (\ie, redundant executions). Second,
they do not incorporate complex multiple fault events (\eg, crashes, reboots)
into their exploration strategies, as such inclusion would exacerbate the
state-space explosion problem.

To address these limitations, I built Semantic-Aware Model Checking
(\textbf{SAMC})
\cite{Leesatapornwongsa+15-SamcIssta,Leesatapornwongsa+14-Samc}, a novel
white-box model checking approach that takes \textit{semantic knowledge} of how
distributed events (specifically, messages, crashes, and reboots) are processed
by the target system and incorporates that information in reduction policies.
The policies are based on sound reduction techniques such as DPOR and symmetry.
The policies tell SAMC not to re-order some pairs of events such as
message-message pairs or crash-message pairs, yet preserves soundness, because
those cut out re-orderings are redundant.

I built SAMC from scratch with 10 KLOC and I was able to reproduce 12 old bugs
in 3 cloud systems involving 30-120 distributed events and multiple crashes and
reboots. Some of these bugs cannot be unearthed by non-SAMC approaches, even
after two days. SAMC can find the bugs up to 271x (33x on average) faster
compared to state-of-the-art techniques. And I found two new bugs in Hadoop
MapReduce and ZooKeeper.

\subsection{Full Semantic-Aware Model Checking (Ongoing Work)} 

There are two major gaps between exisitng dmcks (including SAMC) and real-world
DC bugs. First, dmcks reorder messages by default, but they do not control the
timings of all types of events necessary to unearth DC bugs. For example, MaceMC
and SAMC do not intercept local computation and do not exercise timeouts;
\modist\ and Demeter do not inject multiple crash and reboot timings; and none
of the above include other faults such as untimely disk faults.

Second, controlling all necessary events is technically doable, but it will
``blow up'' the exploration space. The use of semantic relationships between
multiple events such as message-message and crash-message semantics in SAMC can
remove redundant re-orderings. However, more innovations are needed to devise
fast exploration strategies that leverage semantic relationships among all
necessary events.

To address the incompleteness of SAMC, I am building \fullcheck, a dmck that
intercepts all types of necessary events to unearth real-world DC bugs, but
will do so in a fast and scalable manner with the incorporation of semantic
relationships between the events.

% I could talk about other reduction technique here, maybe Demeter, bounded
% model checking from MSR, etc.

\subsection{Automated Semantic-Aware Model Checking (Ongoing Work)} 

So far, as we leverage domain-specific semantic information into reduction
strategies, we (or the developers) must manually extract and incorporate the
semantic knowledge and write the corresponding reduction policies. This manual
process is based on high-level human understanding of the codebase, which can
potentially miss important patterns due to human errors, and breaks soundness.

To address the unsoundness of SAMC, and the developers' burden in manually
writing semantic-based reduction strategies, I am evolving it into \autocheck,
a dmck that automatically and soundly extracts complete semantic knowledge into
reduction strategies with the help of program analysis.
%
To do so, I combine symbolic execution and dmck. While others have used symbolic
execution with model checking for LC bugs, \autocheck\ will be the first case for
implementation-level dmck. 

\subsection{Deep Semantic-Aware Model Checking (Future Work)}

Execution paths to DC bugs often requires complex input preconditions such as
multiple faults, reboots, and protocol initiations. I found that more than 60\%
of DC bugs require more than one protocol initiations, 35\% require multiple
faults, and 29\% of DC bugs arise within buggy interactions between foreground
and background protocols.  This again highlights the complexity of fully
complete systems. If we do not include the complex preconditions, the bugs will
not surface.

To address this complexity, I will construct \deepcheck, a dmck with a backward
static analysis tool that is capable of searching the necessary input
preconditions to cover unreachable paths. The concept of \deepcheck\ is the dmck
will run with a limited input precondition. Then, \deepcheck\ will analyze which
code path is not reachable given the limited input. It will perform a backward
analysis to search for input preconditions to the path. As a result, this
backward analysis will provide the sequence of input preconditions that cover
more complex scenarios.


%\newpage

%\setcounter{page}{1}

%\thispagestyle{empty}

%\singlespacing

%{\small

   % \bibliographystyle{abbrv} % just last name

   % \bibliographystyle{unsrt} % sorted based on appereance

   \bibliographystyle{plain}

   % do not change local[dot]bib because it will be replaced with local-projName.bib
   \bibliography{bibs/defs,bibs/all,bibs/personal,local,bibs/confs}

   %\bibliography{H/defs,H/all,H/personal,local-dcrugs.bib,H/confs}

%}



\end{document}
